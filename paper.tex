\documentclass{article}

\usepackage{amsmath}
\usepackage{amsfonts}
\usepackage{algorithm2e}
\usepackage{graphicx}
\usepackage[font={small,it}]{caption}

\graphicspath{{./gfx/}}

\linespread{1.3}

\begin{document}

\title{Implementation Design Patters for Pursuit-Evasion Problems with Bounded Speed}
\author{Daniel Savage\\CSC 426\\{V00701453}\\savaged@uvic.ca}
\date{November 2012}
\maketitle
\pagebreak

\section{Introduction}
% Explanation of pursuit-evasion problem in general.
% The two various methods for analyzing the problem
% a solution for blending the two solutions together

\section{Paradigm 1 - Discrete Formulation}
The Discrete Formulation of Pursuit-Evasion studies focus on recognizing scenarios in a set of states on a connected graph. This creates a paradigm where strategies are much more easily recognized and explained. Routes and movements are generalized and win/lose situations for pursuers and evaders can often be predicted ahead of time by the nature of the graph played on.

\begin{figure}[htb]
\centering
\includegraphics[width=0.4\textwidth]{"graph1"}
\caption{An example of a Discrete Formulation on a closed graph; evaders and pursuers are marked accordingly as \(P_1,P_2,E_1,E_2,E_3\)}
\end{figure}

Let the environment that the pursuers and evaders coexist in be a finite, connected graph \(G\). Let \(G\)'s vertices be represented as \(v_1,v_2,...,v_n\), each being a distinct location in the environment. All edges are considered to be undirected and of equal length. Let the set of all \(q\) pursuers in \(G\) can be represented as \(P = {p_1,...,p_q}\), with each being on some \(v_i\) in \(G\). Subsequently, also let the set of all \(w\) evaders in \(G\) can be represented as \(E = {e_1,...,e_w}\), with each being on some \(v_i\) in \(G\). Any vertex in \(G\) can be occupied by more than one pursuer and/or evader.

\begin{figure}[htb]
\centering
\includegraphics[width=0.7\textwidth]{"graph2_steps"}
\caption{An example scenario on a graph with three steps where the pursuer wins. Note that in this example the evader \(E_1\) must move each turn. If \(E_1\) is able to "stay put" on a turn it can evade \(P_1\) indefinitely}
\end{figure}

In the Discrete Formulation of Pursuit-Evasion problems, scenarios are usually played to the following rules: \(P\) and \(R\) occupy some vertex of \(G\). Subsequently each alternates moving from one adjacent node to another. This alternation can be from one member of each at a time, or all of one's and then another's. The pursuers win the scenario if they are able to "catch" each evader by occupying the same vertex. Evader(s) do not all need to be caught in the same turn. Alternatively, the evader(s) win if at least one of them can evade the pursuer(s) indefinitely.\cite{copsRobbers}

There can be varieties on each of the rules for this scenario. The Pursuer(s) and/or evader(s) may not need to necessarily move each turn, which can sometimes determine whether or not one side wins or loses. Also the pursuer(s) or evader(s) may not have perfect information when making decisions, only able to "see" others at certain distances. If pursuers do not know any information other than the node they occupy, we can often represent these scenarios as search games. \cite{parsons}

Provided that \(G\) is a planar graph, we can make the assumption that we will only need at most three pursuers to catch all evaders, as proven by Aigner and Fromme. This is performed by one to three pursuers sequentially closing off the region occupied by an evader, tackling each one at a time. One pursuer is able to step into the area where the evader is free to move, and its cohorts (if any) are able to cover its congruent spaces, taking advantage of the planar properties of the graph. This is repeated until each evader is caught. \cite{copsRobbers}

\section{Paradigm 2 - Formulation on the Continuous Plane}
Whilst the Discrete Formulation offers a powerful strategic methodology, it does not sadly reflect a realistic situation very well for a group of pursuers. Movements are not made in instantaneous, distinguishable jumps in reality, nor is Pursuit-Evasion generally a perfect-information game. Formulation on a Continuous Plane, or Continuous Formulation, is the idea behind expressing a Pursuit-Evasion scenario as a geometric real-world model. Concepts for applications of this model often include teams of robots searching for people in an enclosed space.

To start, we will begin by defining the scenario environment as an open set \(R\subset\mathbb{R}^2\). \(R\) is to be a simply-connected and has some sort of polygonal boundary. Some scenarios define that \(R\) has holes. \(P\subset R\) is assumed to be the set of points that represent the locations of the various pursuers of the scenario. Accordingly \(E\subset R\) is assumed to make up the set of evaders, their locations being represented as points as well. Members of \(P\) has perfect information about their location with respect to \(R\) and each other, but not necessarily \(E\). Both pursuers and evaders move at various speeds within \(R\). Members of \(P\) generally are expected to move at a bounded consistent speed \(V_P\), while evaders can be viewed as having either a bounded speed or in some cases, unbounded. \cite{limVis, robotics} Pursuers are generally considered to have "caught" the evaders when being within a certain distance of them.

As stated previously, the set of pursuers \(P\) do not have perfect information of \(R\) and can only see a combined limited region at one time. Subsequently, they may not know where the evaders \(E\) are located. This visible region is known as the \emph{visibility polygon} \(V\). \(V\) is composed of a collection of line segments that may partially align with \(R\)'s shape itself. It's assumed \(V\)'s topology changes changes as it's corresponding \(P\) moves about \(R\). The line segments of \(V\) are that cross across the face of \(R\)'s polygonal shape are known as \emph{gap edges}. These gap edges allow us to divide \(R\) into shrinking and growing regions of visibility. \cite{limVis, robotics}

To map probability of where evaders may be in \(R\), regions currently not visible are given a status of either \emph{contaminated} or \emph{uncontaminated}. A contaminated space represents the idea that an evader may be hiding within it, and the pursuer(s) can not be certain that the area has been completely checked. When gap edges shrink and change position, regions can split and merge together. If a contaminated region merges with an uncontaminated region, the area is said to be \emph{recontaminated} and needs to be re-checked by the pursuer(s). This is to account for evaders potentially moving from one area outside of \(V\) to another.

\section{Blended Paradigm Concept}
% explanation of the paradigm (NP-hardness aleviation)
The aim of this section is to present a paradigm that attempts to combine both advantages of the Discrete and Continuous Formulations for pursuers in a real-time environment. We'll refer to this as the \emph{Blended Paradigm} from now on. The Blended Paradigm works on the idea of having a Discrete Formulation model stored, and update it's state according to observed stimuli found in the Continuous Formulation. Subsequently we make strategic decisions for pursuers in the Discrete Model, and execute them in the Continuous Formulation. This allows us to take advantage of the Discrete Formulation's simplified structure and the Continuous Formulation's real-life applications.

% mathematical definitions and/or data structures
To start, we will begin by defining the scenario environment as an open set \(R\subset\mathbb{R}^2\). \(R\) is to be a simply-connected and has some sort of polygonal boundary. Let sets \(P,E\subset{R}\) be the positions of the pursuers and evaders and their velocities identical to that found in the Continuous Formulation. Let \(G\) be an undirected, weighted-edge graph representing \(R\), with each of its vertex in a state being either \emph{visible, uncontaminated,} or \emph{contaminated}. Later in this section we will explain how \(G\) represents \(R\). Each pursuer and evader will keep track of which vertex it currently "occupies". It is also assumed that all pursuers in \(P\) know their location relative to \(R\).

% delunay triangulation and minimum spanning tree connection
To have \(G\) represent \(R\), we take advantage of \(R\)'s shape being a closed polygon. By performing, a Delaunay Triangulation on \(R\), we acquire a triangulation composed of \(R\)'s vertices. \cite{dTri} It is assumed that the triangles outside of \(R\) are "stripped" and discarded immediately. We are left with a set of adjacent triangles that make up \(R\). For each triangle, add a corresponding vertex to \(G\) and add an edge between each adjacent triangle. The weight of an edge between two triangle's vertices is the distance between each triangle's midpoint in barycentric coordinates. \cite{bCenter} \(G\) now represents a set of general areas in \(R\) and the approximate distances between each.

%sexy diagram

A running scenario for the Blended Paradigm looks incredibly similar to the Continuous Formulation. The main difference is that instead of managing sets of changing visibility polygons, visibility is generalized into triangular sections which are managed in \(G\). Let is be assumed for a pursuer to "catch" an evader, they must both be inside the same triangulated region. In real-time, a pursuer checks at each moment which triangulated regions it can completely see. If a region can be completely seen in \(R\) it is marked as visible in \(G\). A triangulated region that was once visible in \(R\) but subsequently no longer is considered uncontaminated. Regions that have never been visible are considered contaminated. 

Contaminated triangles and vertices in \(G\) are meant to represent regions in \(R\) where an evader could possibly be. In that case, if a contaminated region is adjacent to an uncontaminated region, it is possible that an evader could move from the contaminated to the uncontaminated region, making both uncontaminated. This creates a degree of "spreading" contamination that needs to be kept in check by the pursuers. Since the evaders move at a bounded speed though, it can be expected that contamination does not happen necessarily instantaneously. For this reason the edges in \(G\) have a weight associated with them. Contamination from one vertex/triangle to another happens at a speed: \begin{displaymath}
foo penis
\end{displaymath}

% tesslation through barycenters
% strategic issues

\section{Conclusion}
% no idea what you're goona write here

\section{References}

\begin{thebibliography}{9}

\bibitem{dTri}
  \emph{Delaunay Triangulation},
  Wolfram Mathworld,
  Retrieved November 2012
  
\bibitem{bCenter}
  \emph{Barycentric Coordinates},
  Wolfram Mathworld,
  Retrieved November 2012

\bibitem{copsRobbers}
  M. Aigner and M. Fromme,
  \emph{A Game of Cops and Robbers}.
  Math. Institut, Freie Universitat Berlin,
  Discrete Applied Mathematics, Volume 8, Issue 4,
  1984
  
\bibitem{parsons}
  T.D. Parsons,
  \emph{Pursuit-Evasion in a Graph}.
  The Pennsylvania State University
  Theory and Applications of Graphs, Volume 642, pp 426-441
  1978
  
\bibitem{robotics}
  Benjamin Tovar and Steven M. LaValle,
  \emph{Visibility-based Pursuit-Evasion with Bounded Speed}.
  Department of Computer Science, University of Illinois,
  The International Journal of Robotics Research 2008 27: 1350,
  2008
  
\bibitem{limVis}
  Brian P. Gerkey and Sebastian Thrun and Geoff Gordon,
  \emph{Visibility-based Pursuit-Evasion with limited field of view}.
  Artificial Intelligence Lab, Stanford University,
  The National Conference on Artificial Intelligence 2004,
  2004

\end{thebibliography}

\end{document}
